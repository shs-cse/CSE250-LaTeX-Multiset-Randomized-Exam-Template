% marks instead of points
\marksnotpoints
% use first brackets for \subpart
\renewcommand{\subpartlabel}{(\thesubpart)}
% use red bold text to show \CorrectChoice
\CorrectChoiceEmphasis{\color{red}\bfseries\boldmath}
% use circled uppercase letters for choices
\newcommand*\circled[1]{\tikz[baseline=(char.base)]{
            \node[shape=circle,draw,inner sep=2pt] (char) {#1};}}
\renewcommand{\choicelabel}{\circled{{\thechoice}}}
% formatting of Question blocks
\qformat{
    \parbox[][24pt][t]{\columnwidth}{ % for creating space below
        \Large\textbf{
            $\diamondsuit$
            Question \thesetquestion~of \numqinrange{\set}\;
            \textit{%
                \hfill%
                \ifthenelse{\equal{\thequestiontitle}{\thequestion}}%
                    {}%
                    {[\thequestiontitle]\;}%
                [\ifthenelse{\equal{\totalpoints}{0}}{%
                        \marks\totalbonuspoints\;(bonus)%
                    }{%
                        \marks\totalpoints%
                        \ifthenelse{\equal{\totalbonuspoints}{0}}
                        {}
                        {\,+\,\marks\totalbonuspoints\;(bonus)}%
                    }%
                ]%
            }%
        }%
    }%
}
% formatting of Bonus Question blocks
\bonusqformat{
    \parbox[][24pt][t]{\columnwidth}{ % for creating space below
        \Large\textbf{
            $\diamondsuit$
            Question \thesetquestion~of \numqinrange{\set}\;
            \textit{%
                \hfill%
                \ifthenelse{\equal{\thequestiontitle}{\thequestion}}
                    {}
                    {[\thequestiontitle]\;}
                [\marks\totalbonuspoints\;(bonus)]%
            }%
        }%
    }%
}
% formatting of subpart, subsubpart etc.
\pointformat{\small\textbf{\textit{[\thepoints]}}}
% formatting of bonussubpart, bonussubsubpart etc.
\bonuspointformat{\small\textbf{\textit{[\thepoints]}}}